% The New Quill Developer's Manual
% Copyright (C) 2025 Jorge Fuertes Alfranca
% Released under the GPL v3 or later licenses
%

\documentclass{report}

% Packages used in this example
\usepackage{graphicx}  % for including images and pdf graphics
\usepackage{microtype} % for typographical enhancements
\usepackage{hyperref}  % for hyperlinks
\usepackage[spanish]{babel} % spanish language
\usepackage[a4paper,top=4.2cm,bottom=4.2cm,left=3.5cm,right=3.5cm]{geometry} % for setting page size and margins

% config
\setlength{\parskip}{1em}

% macros
\newcommand{\tnq}{\textit{The New Quill}}
\newcommand{\ic}[1]{\texttt{#1}}
\newcommand{\la}[1]{\textit{#1}}

% front page
\title{\tnq\\ \large Manual del escritor}
\author{\copyright\ 2025 Jorge Fuertes Alfranca \\ \small Released under the GPL v3 or later licenses}

\begin{document}
\begin{center}
\includegraphics[width=0.98\textwidth]{gfx/portada.pdf}
\end{center}
\newpage

\maketitle
\tableofcontents
\newpage

\chapter{Bienvenida}
    \section{Introducción}
        Bienvenido a \tnq, un sistema de desarrollo de aventuras conversacionales para ordenadores modernos.
        
        Este software están inspirado en los sistemas de desarrollo de aventuras conversaciones de los ochenta,
        especialmente en los de \textit{Gillsoft}, como el programa \textit{PAWS} de \textit{ZX Spectrum} y el
        compilador \textit{DAAD} de PC, además de en todo el trabajo hecho por los pioneros de las aventuras
        conversacionales, con mención especial a la empresa española \textit{Aventuras AD}.

\chapter{El compilador}        
    \section{Línea de comandos}
    \section{Identificadores y etiquetas}


\chapter{Lenguaje de programación}
    \section{Estructura del programa}
        La sección de procesos empieza, en nuestros ficheros fuente, con la directiva \ic{SECTION PROCESS TABLES}.
        Dentro de esta sección, definimos cada tabla de procesos con la directiva \ic{TABLE <etiqueta>}, y la
        cerramos con la directiva \ic{END TABLE}.

        La etiqueta es única, es decir, no puede haber dos tablas con la misma etiqueta en el mismo programa.

        \subsection{Etiquetas reservadas}
            Hay ciertas etiquetas reservadas, que pueden y deben usarse para definir tablas de procesos, pero que tienen
            un significado especial para el sistema. Estas etiquetas son:
            \begin{itemize}
                \item \ic{init}: Tabla de procesos de inicialización del juego. Se ejecuta al inicio de la aventura,
                    toda la tabla basándonos en las condiciones.
                \item \ic{response}: Tabla de respuestas en la que se buscan las sentencias lógicas. Se ejecuta cuando
                    el jugador introduce un comando. Se ejecutan los procesos que correspondan con
                    \ic{<verbo\_1>} \ic{<nombre\_1>}.
                \item \ic{turn}: Tabla de procesos que se ejecuta en cada turno del juego, justo antes de la tabla de
                    respuestas. Se procesan todas las acciones basándose en las condiciones.
                \item \ic{location}: Tabla de procesos que se ejecuta al entrar en una nueva localidad o al pedir
                    una descripción de la misma. El programador es responsable de comprobar la localidad con una
                    condición, o bien no hacerlo y asumir que esa parte de la tabla se ejecuta siempre para todas
                    las localidades.
                \item \ic{cron}: Tablas que son procesadas en segundo plano. Cada proceso se define con
                    \ic{EVERY <n> <turns|hour|min|sec>}, y se ejecuta cada n turnos, horas, minutos o segundos. La
                    ejecución se bloquea temporalmente si se está ejecutando cualquier otra tabla, y viceversa, no se
                    entra, por ejemplo, en la tabla de respuestas si se está ejecutando un proceso cron hasta que este
                    termina.
            \end{itemize}

    \section{Tipos de datos}
    \section{Variables}
    \section{Constantes}
    \section{Operadores}
    \section{Estructuras de control}
    \section{Funciones y procedimientos}
    \section{Bibliotecas estándar}

\end{document}